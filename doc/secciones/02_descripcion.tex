\chapter{Problem description}
\section{The necessity of a distributed system for collaborative science}
Nowadays, computers play an essential role in scientific research. For most scientists, a desktop machine is enough to store and process the data they work with, but a considerable amount of them need supercomputers to be able to deal with more complex problems \cite{computing-in-science}.

Supercomputers are expensive, and price ranges vary widely depending on the computational capacities of the system; For example, a supercomputer with a power of 40 TFLOPS like \textit{Alhambra}, the supercomputer at the \textit{University of Granada}, costs around \$670,000\cite{ideal-alhambra}. A supercomputer leading the list Top500,  like the one from \textit{Oak Ridge National Laboratory}, with 200 PetaFLOPS manufactured by IBM costs \$200 million\cite{oak-ridge}. Hence having access to one of these machines is not common.

Distributed systems are an attractive alternative to supercomputers. A distributed system can be defined as a set of independent machines that interact with each other, cooperating to achieve a common objective. Even though this approach is much cheaper than using a supercomputer, an important investment is required in order to buy and set up these machines.

In 2002, the \textit{University of California, Berkeley} addressed the problem described above by developing \textit{BOINC}\cite{boinc-website}, a platform for volunteer computing where users can contribute to different scientific projects with their computers or smart-phones. At the moment this chapter was written, the platform has achieved a power of over 27 PetaFLOPS in the last twenty-four hours by using 563,506 computers provided by 142,911 volunteers.

Even though \textit{BOINC} is a successful example of how volunteer computing is possible and what it can achieve, it still has room for improvement that this project aims to provide. 

On the one hand, a user willing to volunteer on \textit{BOINC} needs to download and set up his own machine in order to start collaborating with a project: non-tech-savvy users can struggle with this and may give up soon. This project addresses this problem by providing users a way to collaborate where they do not need to install anything apart from a web browser. Since most (if not all) consumer-oriented operative systems such as \textit{Ubuntu}, \textit{Mac OSX} or \textit{Windows} come with a web-browser installed out of the box, this platform provides a zero-installation method for volunteer computing.

On the other hand, \textit{BOINC} uses a Client-Server architecture where user's machines communicate with a centralized server that assigns them tasks. This centralized approach has a scalability problem where in case that the number of volunteers increases extremely or the amount of requests per user is high enough, the central server will not be able to handle the number of requests made by a user and therefore the performance of the system will reach its limit. This approach makes the system to have a single point of failure as well; if the central server goes down, users will not be able to work towards the common objective since they cannot communicate with each other, and therefore the whole experiment will be paused until the central server goes alive again.  

By using a decentralized architecture where nodes that satisfy certain criteria (detailed in later chapters) act as coordinators of the system, this project aims to provide better scalability and fault-tolerance than BOINC. 


\section{Evolutionary Algorithms}
Evolutionary Algorithms are metaheuristic optimization algorithms that take natural evolution as an inspiration to solve problems. They apply bio-inspired operations known as \textit{genetic operators} over a set of solutions to the problem called \textit{population}. Each solution inside the \textit{population} is known as \textit{chromosome} and has a \textit{fitness} value based on the quality of the solution itself. A \textit{chromosome} is made of \textit{genes}, these genes are specific characteristics of a solution representation. 
 
The simulation of this natural process is a probabilistic optimization technique that frequently achieves better results than other classic methods of solving hard problems. \cite{galist}. 

An overview of the implementation of this algorithm consists of the following steps

\begin{enumerate}
    \item \textbf{Initialize:} create a population with randomly generated chromosomes, and evaluate the fitness of each one.
    
    \item Until an specified termination condition is meet, apply the following genetic operators over the population:
    
    \begin{enumerate}
        \item \textbf{\textit{Selection}:} pick which chromosomes from the population will reproduce, in other words, the \textit{crossover} operator will be applied. We should guarantee that the best chromosomes in the population have higher likelihood to be selected, while still giving the opportunity to chromosomes with less quality to reproduce.
    
        \item \textbf{\textit{Crossover}:} given two chromosomes, this operator will generate two or more new solutions that will inherit genes from both parents but will be different from them.
        
        \item \textbf{\textit{Mutation}:} with a given probability, choose chromosomes to modify some of their genes.
        
        \item \textbf{\textit{Evaluation}:} evaluate the fitness of each chromosome in the population.
        
        \item \textbf{\textit{Replacement}:} replace the original population with the new one, following a given criteria. 
        
    \end{enumerate}     
\end{enumerate}

Selection, crossover and replacement are explotation methods. Mutation and Initialization are diversification methods.

As we can see, there are some parameters here we can play with: population size, probabilities, etc...


\section{Distributed Systems}

\section{Technosocial Systems}
